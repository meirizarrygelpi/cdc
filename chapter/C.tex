% Copyright 2018 Melvin Eloy Irizarry-Gelpí
\chapter{W Binions}
%%%%%%%%%%%%%%%%%%%%%%%%%%%%%%%%%%%%%%%%%%%%%%%%%%%%%%%%%%%%%%%%%%%%%%%%%%%%%%%%
A W binion has the form
\begin{equation}
    a_{0} + a_{1} W
\end{equation}
These follow from a parabolic Cayley-Dickson construct on the reals. Since you need two real numbers for a W binion, you can represent a W binion as a pair.
%%%%%%%%%%%%%%%%%%%%%%%%%%%%%%%%%%%%%%%%%%%%%%%%%%%%%%%%%%%%%%%%%%%%%%%%%%%%%%%%
\section{Multiplication}
%%%%%%%%%%%%%%%%%%%%%%%%%%%%%%%%%%%%%%%%%%%%%%%%%%%%%%%%%%%%%%%%%%%%%%%%%%%%%%%%
The multiplication operation is non-trivial. Since the W binions are a parabolic Cayley-Dickson construct, you use the parabolic multiplication operation. If
\begin{align}
    x &= \begin{pmatrix}
        x_{0} & x_{1}
    \end{pmatrix} & y &= \begin{pmatrix}
        y_{0} & y_{1}
    \end{pmatrix}
\end{align}
then
\begin{equation}
    z = x \wedge y = \begin{pmatrix}
        x_{0} y_{0} & y_{1} x_{0} + y_{0} x_{1}
    \end{pmatrix}
\end{equation}
Note that this multiplication operation is commutative. Note that
\begin{equation}
    W = \begin{pmatrix}
        0 & 1
    \end{pmatrix}
\end{equation}
which leads to
\begin{equation}
    W \wedge W = \begin{pmatrix}
        0 & 0
    \end{pmatrix}
\end{equation}
That is, $W$ is nilpotent.
%%%%%%%%%%%%%%%%%%%%%%%%%%%%%%%%%%%%%%%%%%%%%%%%%%%%%%%%%%%%%%%%%%%%%%%%%%%%%%%%
\section{Conjugate Operations}
%%%%%%%%%%%%%%%%%%%%%%%%%%%%%%%%%%%%%%%%%%%%%%%%%%%%%%%%%%%%%%%%%%%%%%%%%%%%%%%%
There are many conjugate operations acting on W binions.
%%%%%%%%%%%%%%%%%%%%%%%%%%%%%%%%%%%%%%%%%%%%%%%%%%%%%%%%%%%%%%%%%%%%%%%%%%%%%%%%
\subsection{Asterisk Conjugation}
%%%%%%%%%%%%%%%%%%%%%%%%%%%%%%%%%%%%%%%%%%%%%%%%%%%%%%%%%%%%%%%%%%%%%%%%%%%%%%%%
If
\begin{equation}
    z = \begin{pmatrix}
        z_{0} & z_{1}
    \end{pmatrix}
\end{equation}
then the asterisk conjugation is
\begin{equation}
    {\ast z} = \begin{pmatrix}
        z_{0} & -z_{1}
    \end{pmatrix}
\end{equation}
Note that this is an involution.
%%%%%%%%%%%%%%%%%%%%%%%%%%%%%%%%%%%%%%%%%%%%%%%%%%%%%%%%%%%%%%%%%%%%%%%%%%%%%%%%
\subsection{Quadrance}
%%%%%%%%%%%%%%%%%%%%%%%%%%%%%%%%%%%%%%%%%%%%%%%%%%%%%%%%%%%%%%%%%%%%%%%%%%%%%%%%
If
\begin{equation}
    z = \begin{pmatrix}
        z_{0} & z_{1}
    \end{pmatrix}
\end{equation}
then the quadrance is given by
\begin{equation}
    \Vert z \Vert^{2} = (z_{0})^{2}
\end{equation}
Note that the quadrance takes non-negative values. However, you can have zero-quadrance from a non-trivial W binion. Any W binion of the form
\begin{equation}
    \begin{pmatrix}
        0 & z_{1}
    \end{pmatrix}
\end{equation}
has zero quadrance.
%%%%%%%%%%%%%%%%%%%%%%%%%%%%%%%%%%%%%%%%%%%%%%%%%%%%%%%%%%%%%%%%%%%%%%%%%%%%%%%%
\subsection{Cloak Conjugation}
%%%%%%%%%%%%%%%%%%%%%%%%%%%%%%%%%%%%%%%%%%%%%%%%%%%%%%%%%%%%%%%%%%%%%%%%%%%%%%%%
If
\begin{equation}
    z = \begin{pmatrix}
        z_{0} & z_{1}
    \end{pmatrix}
\end{equation}
then the cloak conjugation is
\begin{equation}
    {\diamond z} = \begin{pmatrix}
        -z_{0} & z_{1}
    \end{pmatrix}
\end{equation}
Note that this is an involution. Cloak conjugation is equivalent to negation of the asterisk conjugation.
%%%%%%%%%%%%%%%%%%%%%%%%%%%%%%%%%%%%%%%%%%%%%%%%%%%%%%%%%%%%%%%%%%%%%%%%%%%%%%%%
\subsection{Dagger Conjugation}
%%%%%%%%%%%%%%%%%%%%%%%%%%%%%%%%%%%%%%%%%%%%%%%%%%%%%%%%%%%%%%%%%%%%%%%%%%%%%%%%
If
\begin{equation}
    z = \begin{pmatrix}
        z_{0} & z_{1}
    \end{pmatrix}
\end{equation}
then the dagger conjugation is
\begin{equation}
    {\dagger z} = \begin{pmatrix}
        z_{0} & -z_{1}
    \end{pmatrix}
\end{equation}
Note that this is equivalent to the asterisk conjugation.
%%%%%%%%%%%%%%%%%%%%%%%%%%%%%%%%%%%%%%%%%%%%%%%%%%%%%%%%%%%%%%%%%%%%%%%%%%%%%%%%
\subsection{Hodge Star}
%%%%%%%%%%%%%%%%%%%%%%%%%%%%%%%%%%%%%%%%%%%%%%%%%%%%%%%%%%%%%%%%%%%%%%%%%%%%%%%%
If
\begin{equation}
    z = \begin{pmatrix}
        z_{0} & z_{1}
    \end{pmatrix}
\end{equation}
then the Hodge star is
\begin{equation}
    {\star z} = \begin{pmatrix}
        z_{1} & z_{0}
    \end{pmatrix}
\end{equation}
Note that this is an involution.
%%%%%%%%%%%%%%%%%%%%%%%%%%%%%%%%%%%%%%%%%%%%%%%%%%%%%%%%%%%%%%%%%%%%%%%%%%%%%%%%
\section{Zero-Divisors}
%%%%%%%%%%%%%%%%%%%%%%%%%%%%%%%%%%%%%%%%%%%%%%%%%%%%%%%%%%%%%%%%%%%%%%%%%%%%%%%%
Since multiplication of non-zero W binions can lead to zero, there are non-trivial zero-divisors. For example, if
\begin{align}
    x &= \begin{pmatrix}
        0 & x_{1}
    \end{pmatrix} & y &= \begin{pmatrix}
        0 & y_{1}
    \end{pmatrix}
\end{align}
then
\begin{equation}
    z = x \wedge y = \begin{pmatrix}
        0 & 0
    \end{pmatrix}
\end{equation}
%%%%%%%%%%%%%%%%%%%%%%%%%%%%%%%%%%%%%%%%%%%%%%%%%%%%%%%%%%%%%%%%%%%%%%%%%%%%%%%%
\section{Decompositions}
%%%%%%%%%%%%%%%%%%%%%%%%%%%%%%%%%%%%%%%%%%%%%%%%%%%%%%%%%%%%%%%%%%%%%%%%%%%%%%%%
With so many involutions you can have many decomposition.
%%%%%%%%%%%%%%%%%%%%%%%%%%%%%%%%%%%%%%%%%%%%%%%%%%%%%%%%%%%%%%%%%%%%%%%%%%%%%%%%
\subsection{Asterisk Decomposition}
%%%%%%%%%%%%%%%%%%%%%%%%%%%%%%%%%%%%%%%%%%%%%%%%%%%%%%%%%%%%%%%%%%%%%%%%%%%%%%%%
Since the asterisk conjugation is an involution, you can have the following decomposition:
\begin{equation}
    z = \frac{1}{2} \left( z + {\ast z} \right) + \frac{1}{2} \left( z - {\ast z} \right)
\end{equation}
The first term is the self-asterisk part of $z$ and the second term is the anti-self-asterisk part of $z$. If
\begin{equation}
    z = \begin{pmatrix}
        z_{0} & z_{1}
    \end{pmatrix}
\end{equation}
then the \textbf{self-asterisk part} of $z$ is given by
\begin{equation}
    \frac{1}{2} \left( z + {\ast z} \right) = \begin{pmatrix}
        z_{0} & 0
    \end{pmatrix}
\end{equation}
and the \textbf{anti-self-asterisk part} of $z$ is given by
\begin{equation}
    \frac{1}{2} \left( z - {\ast z} \right) = \begin{pmatrix}
        0 & z_{1}
    \end{pmatrix}
\end{equation}
Note that the anti-self-asterisk part of $z$ is always nilpotent.
%%%%%%%%%%%%%%%%%%%%%%%%%%%%%%%%%%%%%%%%%%%%%%%%%%%%%%%%%%%%%%%%%%%%%%%%%%%%%%%%
\subsection{Hodge Star Decomposition}
%%%%%%%%%%%%%%%%%%%%%%%%%%%%%%%%%%%%%%%%%%%%%%%%%%%%%%%%%%%%%%%%%%%%%%%%%%%%%%%%
Since the Hodge star operation is an involution, you can have the following decomposition:
\begin{equation}
    z = \frac{1}{2} \left( z + {\star z} \right) + \frac{1}{2} \left( z - {\star z} \right)
\end{equation}
The first term is the self-star part of $z$ and the second term is the anti-self-star part of $z$. If
\begin{equation}
    z = \begin{pmatrix}
        z_{0} & z_{1}
    \end{pmatrix}
\end{equation}
then the \textbf{self-star part} of $z$ is given by
\begin{equation}
    \frac{1}{2} \left( z + {\star z} \right) = \frac{1}{2} (z_{0} + z_{1}) \begin{pmatrix}
        1 & 1
    \end{pmatrix}
\end{equation}
and the \textbf{anti-self-star part} of $z$ is given by
\begin{equation}
    \frac{1}{2} \left( z - {\star z} \right) = \frac{1}{2} (z_{0} - z_{1}) \begin{pmatrix}
        1 & {-1}
    \end{pmatrix}
\end{equation}
Note that the self-star and anti-self-star parts are proportional to W binions that are idempotent. If
\begin{align}
    P &\equiv \frac{1}{2} \begin{pmatrix}
        1 & 1
    \end{pmatrix} & Q &\equiv \frac{1}{2} \begin{pmatrix}
        1 & {-1}
    \end{pmatrix}
\end{align}
then
\begin{align}
    P \wedge P &= P & Q \wedge Q &= Q
\end{align}
%%%%%%%%%%%%%%%%%%%%%%%%%%%%%%%%%%%%%%%%%%%%%%%%%%%%%%%%%%%%%%%%%%%%%%%%%%%%%%%%
\section{Differential Operators}
%%%%%%%%%%%%%%%%%%%%%%%%%%%%%%%%%%%%%%%%%%%%%%%%%%%%%%%%%%%%%%%%%%%%%%%%%%%%%%%%
With two real variables $x$ and $y$, you can define a W binion variable $z$ via
\begin{equation}
    z = \begin{pmatrix}
        x & y
    \end{pmatrix}
\end{equation}
%%%%%%%%%%%%%%%%%%%%%%%%%%%%%%%%%%%%%%%%%%%%%%%%%%%%%%%%%%%%%%%%%%%%%%%%%%%%%%%%
\section{M\"{o}bius Transformations}
%%%%%%%%%%%%%%%%%%%%%%%%%%%%%%%%%%%%%%%%%%%%%%%%%%%%%%%%%%%%%%%%%%%%%%%%%%%%%%%%
...
%%%%%%%%%%%%%%%%%%%%%%%%%%%%%%%%%%%%%%%%%%%%%%%%%%%%%%%%%%%%%%%%%%%%%%%%%%%%%%%%
\section{Cross-Ratio}
%%%%%%%%%%%%%%%%%%%%%%%%%%%%%%%%%%%%%%%%%%%%%%%%%%%%%%%%%%%%%%%%%%%%%%%%%%%%%%%%
...